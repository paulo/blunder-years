\chapter{Museu da Pessoa — tratamento de fotografias}
\label{cap:museu}
<Pequeno resumo do problema>
\section{Analise e Especificação}
\label{seq:museu-ana}
<O que é preciso fazer de forma tecnica, i.e. temos que apanhar isto e acoloutro guardar mas depois temos que > 
\section{Implementação}
\label{seq:museu-imp}
\subsection{Estrutura de dados}
\label{seq:museu-est}
<estrutura de dados criada>
\subsection{Filtro de Texto}
\label{seq:museu-filtro}
<ficheiro flex>
\subsection{Funcionamento}
\label{seq:museu-func}
<Como o ficheiro flex funciona e chama poe na estrutura de dados>
\section{Testes realizados}
\label{seq:museu-test}
<alguns exemplos>


% respectivo enunciado da descricao do problema, das decisoes que lideraram o desenho da solucao e sua implementacao  (incluir a especificacao Flex , deverao conter exemplos de utilizacao (textos fontes diversos e respectivo resultado produzido).

% conceção e arquitetura do sistema

% testes antes da conclusao e intercalar imagens com o texto

% código em apendice em anexo

% dificuldades e decisões antes do código

% imagens bem descritas por texto

% Problema->decisão->solução

% analise e especificação é depois da descrição do problema

% análise: pega nisto, faz isto
% (descascar o problema)=> resultado(especificação) temos isto e dá isto


% respectivo enunciado da descricao do problema, => A introdução depois do titulo 
% as decisoes que lideraram o desenho da solucao e sua implementacao  => estrutura de dados
% (incluir a especificacao Flex ,                       => filtro de texto
% deverao conter exemplos de utilizacao (textos fontes diversos e respectivo resultado produzido) => exemplos