\documentclass[a4paper,10pt]{report}

\usepackage{graphicx}
\usepackage{color}

\usepackage{caption}
\usepackage{subcaption}

\usepackage[portuguese]{babel}
\usepackage[utf8]{inputenc}
\usepackage[T1]{fontenc}

\usepackage{geometry}
\geometry{a4paper}
\usepackage[parfill]{parskip}

\usepackage{changepage}

\usepackage{amsmath}

\usepackage{fancyhdr}

\usepackage{nopageno}

\graphicspath{{./imagens/}}

\usepackage{url}

\usepackage{verbatim}
\usepackage{fancyvrb}

\usepackage[colorlinks=true,linkcolor=blue,citecolor=blue]{hyperref}

\usepackage{listings}
\renewcommand{\lstlistingname}{Código}
\usepackage{color}
\definecolor{grey}{rgb}{0.9,0.9,0.9}
\definecolor{greyD}{rgb}{0.5,0.5,0.5}

\lstnewenvironment{code}[1][]%
{
   \noindent
   \lstset{
  language=C,
  float=htpb,
  backgroundcolor=\color{grey},
  basicstyle=\scriptsize,
  numbers=left,
  numbersep=5pt,
  numberstyle=\tiny\color{greyD},
  breaklines=true,
  frame=single,
  #1}
}
{}

\begin{document}

\include{capa}
%----------------------------------------------------------------------
%\newpage
%\phantom{placeholder} % doesn't appear on page
%\thispagestyle{empty} % if want no header/footer
%----------------------------------------------------------------------
\tableofcontents
%\phantom{placeholder} % doesn't appear on page
\thispagestyle{empty} % if want no header/footer
%----------------------------------------------------------------------
%\newpage
%\phantom{placeholder} % doesn't appear on page
%\thispagestyle{empty} % if want no header/footer
%----------------------------------------------------------------------
%\pagestyle{fancy}
%\setlength{\headheight}{15.2pt}
%\fancyhf{} % apagar as configurações actuais
%\fancyfoot[LE,RO]{\thepage}
%\fancyhead[LE,RO]{PL - Trabalho Pratico 1 --- Araujo P., Belo O., Oliveira R.}
%\page{setcounter}{0}
%----------------------------------------------------------------------

\chapter{Introdução}
\label{cap:intro}
O presente projeto enquandra-se na unidade curricular de Processamento de Linguagnes do curso de  Licenciatura em Engenharia Informática da Universidade do Minho.
O projeto pretende aumentar as capacidades com as expressões regulares, desenvolvendo processadores de linguagens regulares utilizando o gerador de filtros de texto Flex.
Para isso foram selecionados 3 exercícios dentro de um grupo de 8 exercícios, são eles: \emph{2.1 Museu da Pessoa — tratamento de fotografias},
\emph{2.2 Processamento de Entidades Nomeadas (Enamex)} e \emph{2.5 Processamento de ficheiros com Canções}.

\chapter{Museu da Pessoa — tratamento de fotografias}
\label{cap:museu}
<Pequeno resumo do problema>
\section{Analise e Especificação}
\label{seq:museu-ana}
<O que é preciso fazer de forma tecnica, i.e. temos que apanhar isto e acoloutro guardar mas depois temos que > 
\section{Implementação}
\label{seq:museu-imp}
\subsection{Estrutura de dados}
\label{seq:museu-est}
<estrutura de dados criada>
\subsection{Filtro de Texto}
\label{seq:museu-filtro}
<ficheiro flex>
\subsection{Funcionamento}
\label{seq:museu-func}
<Como o ficheiro flex funciona e chama poe na estrutura de dados>
\section{Testes realizados}
\label{seq:museu-test}
<alguns exemplos>


% respectivo enunciado da descricao do problema, das decisoes que lideraram o desenho da solucao e sua implementacao  (incluir a especificacao Flex , deverao conter exemplos de utilizacao (textos fontes diversos e respectivo resultado produzido).

% conceção e arquitetura do sistema

% testes antes da conclusao e intercalar imagens com o texto

% código em apendice em anexo

% dificuldades e decisões antes do código

% imagens bem descritas por texto

% Problema->decisão->solução

% analise e especificação é depois da descrição do problema

% análise: pega nisto, faz isto
% (descascar o problema)=> resultado(especificação) temos isto e dá isto


% respectivo enunciado da descricao do problema, => A introdução depois do titulo 
% as decisoes que lideraram o desenho da solucao e sua implementacao  => estrutura de dados
% (incluir a especificacao Flex ,                       => filtro de texto
% deverao conter exemplos de utilizacao (textos fontes diversos e respectivo resultado produzido) => exemplos
\chapter{Processamento de Entidades Nomeadas}
\label{cap:enamex}

Entidades nomeadas são "elementos atómicos em texto" pertencentes a categorias predefinidas tais como nomes de pessoas, organizações, localizações, quantidades, etc. Assim, Processamento de Entidades Nomeadas(PEN) é a tarefa de identificar estas entidades.
Embora as categorias das entidades nomeadas serem predefinidas, existem várias opiniões sobre que categorias deve ser consideradas entidades nomeadas e quão abrangentes estas categorias devem ser. Por convenção, tags \emph{"ENAMEX"} são utilizadas para nomes, tags \emph{"NUMEX"} são utilizadas para entidades numéricas, e tags \emph{"TIMEX"} são utilizadas para entidades temporais.

Neste exercício iremos apenas processar entidades com a tag \emph{"ENAMEX"}, na forma:
\begin{itemize}
\item\verb!<ENAMEX TYPE="PERSON">Francisco de Vilela Barbosa</ENAMEX>!\\(Pessoa)
\item\verb!<ENAMEX TYPE="LOCATION" SUBTYPE="COUNTRY">Portugal</ENAMEX>!\\ (Localização, País)
\item\verb!<ENAMEX TYPE="LOCATION" SUBTYPE="CITY">Rio de Janeiro</ENAMEX>!\\  (Localização, Cidade)
\item\verb!<ENAMEX TYPE="ORGANIZATION">Universidade do Minho</ENAMEX>!\\(Organização)
\end{itemize}

Como exercício extra iremos também abordar as entidades na forma:
\begin{itemize}
\item\verb!<ENAMEX TYPE="LOCATION"> Santo Novo </ENAMEX>!\\(Localização não específica)
\end{itemize}
Todas as outras tags irão ser ignoradas.

O processamento de entidades nomeadas, apesar de ser aparentemente uma tarefa simples, enfrenta um dado numero de desafios. As entidades podem tornar-se difíceis de encontrar, e uma vez encontradas, difíceis de classificar. Localizações e nomes de pessoas podem ser as mesmas, e seguir estilos similares de formatação.

\section{Analise e Especificação}
\label{seq:enamex-ana}

Uma breve leitura do problema permite-nos entender algumas das funcionalidades necessárias, sendo estas definidas como:
\begin{itemize}
\item Necessidade de ordenação e não repetição na listagem de pessoas(alínea a): \\
	\begin{itemize}
	\item A alínea A do problema requer a listagem de todas as pessoas identificadas, sem repetições. Este informação refere-se então às tags do tipo:\\
	\verb!<ENAMEX TYPE="PERSON">...</ENAMEX>!.\\
	Por forma a armazenar e ordenar adequadamente toda a informação acerca das Pessoas, é necessária a utilização de uma estrutura capaz de suportar esta informação. 
	\end{itemize}
	
\item Listar os países e cidades marcadas (alínea b): \\
	\begin{itemize}
	\item Apesar de não estar especificado no enunciado, o grupo propôs uma implementação na qual seria possível associar cidades a certos países. Compreendemos que este tipo de implementação, para grande parte dos casos, não é viável e pode tornar a informação apresentada incoerente. No entanto, no intuito de aprender e aumentar o desafio proposto, decidimos que cidades mencionadas após países e antes de pontos finais pertenciam a esses países. Esta informação é também armazenada na estrutura implementada.
	\end{itemize}
\item Listar as organizações (alínea c): \\
	\begin{itemize}
	\item Similarmente à alínea A, a alínea C requer a listagem de todas as organizações identificadas. Esta informação refere-se então às tags do tipo: \\
	\verb! <ENAMEX TYPE="ORGANIZATION">...</ENAMEX>! \\
	E está implementada de forma similar à listagem de pessoas.
	\end{itemize}
\item Apresentar os resultados em formato HTML: \\
	\begin{itemize}
	\item Por forma a visualizar facilmente os resultados do processamento do texto, estes são apresentados em formato HTML através da implementação de funções capazes de transformar a informação contida nas estruturas em documentos de texto com o formato requerido.
	\end{itemize}
\end{itemize}

\section{Implementação}
\label{seq:enamex-imp}
\subsection{Estrutura de dados}
\label{seq:enamex-est}
Com o intuito de cumprir todos os requisitos estruturais definidos anteriormente, foi desenvolvida uma estrutura de dados única capaz de armazenar todos os dados necessários. Assim, escolhemos implementar uma árvore binária de procura, na qual os nodos possuem a informação a guardar sobre a forma de array de caracteres. A escolha desta estrutura facilita a ordenação alfabética dos diversos nomes que possamos processar, e é de implementação relativamente simples. Creemos ser superior a outras estruturas tais como listas ligadas cuja implementação, apesar de mais simples, torna-se mais complexa quando é necessária a ordenação dos seus elementos (O(N)) e a tabelas de hash cuja implementação é mais complexa, sendo que para elevadas quantidades de dados possuem ainda a necessidade de reHashing e garbage colection. 

\begin{figure}[H]
\centering
\includegraphics[width=7cm]{anexos/2-2/tree.png}
\caption{Representação gráfica da estrutura de dados}
\end{figure}

Para casos em que apenas é necessário o armazenamento da entidade, sem qualquer tipo de associação, uma árvore com dois apontadores (esquerda e direita) e com a capacidade de armazenar a informação (array de caracteres) bastaria para abordar todos os casos. No entanto, de forma a armazenar a possível associação entre os países e as suas cidades, modificamos a árvore previamente referida, e adicionamos-lhe um apontador extra, que poderá ser visto como o apontador para a raiz de uma nova árvore, constituída por todas as cidades que pertencem a um dado país. Como curiosidade, adicionamos também um inteiro em cada nodo que servirá como contador para todas as ocorrências de um dado elemento. Todo o código referente a esta estrutura encontra-se implementado no ficheiros "tree.c" e "tree.h" e pode ser consultado em anexo.

\subsection{Filtro de Texto}
\label{seq:enamex-filtro}

De forma a processar as tags pertencentes a pessoas/organizações, foram criadas expressões regulares capazes de identificar essas tags.
Assim, foram implementadas as seguintes expressões regulares para pessoas/organizações, assim como abreviaturas que facilitam a sua leitura e compreensão:

\begin{itemize}
\item pessoas: \verb! {enamex}{person}{pal}{eclose}! 
\item organizações:\verb!{enamex}{org}{pal}{eclose}!
\item localizações gerais:\verb! {enamex}{loc}{pal}{eclose}!
\item cidades: \verb!{enamex}{loc}{subcity}{pal}{eclose}!
\end{itemize}

sendo que as abreviaturas mencionadas correspondem a:

\begin{itemize}
\item Palavra \\
	\verb!    {pal} [a-zA-Z0-9Ç-ÑÀ-û ]+!
\item Tag Enamex \\
	\verb!    {enamex} \<[ \t]*(?i:enamex)[ \t]+(?i:type)=!
\item Tag de fecho \\
	\verb!    {eclose} \<[ \t]*\/(?i:enamex)[ \t]*\>!
\item Elemento "Organization"\\
	\verb!    {org} \"[ \t]*(?i:organization)[ \t]*\"[ \t]*\>!
\item Elemento "Person"\\
	\verb!    {person} \"[ \t]*(?i:person)[ \t]*\"[ \t]*\>!
\end{itemize}


De notar a capacidade das expressões regulares identificarem tags definidas tanto em letra maiuscula como minuscula, sendo também tolerantes à quantidade de espaços ou tabs presentes entre elementos destas.
De forma a implementar a capacidade de associar cidades a um dado país, foram utilizados "operadores de contexto", de forma a que caso seja detectado uma tag correspondente a um país, o analisador entre no contexto não exclusivo (\%s)country, e associa as seguintes tags correspondentes a cidades ao país em causa. Caso seja detetado um ponto final, o analisador lexico abandona esse contexto e continua a processar em contexto geral. 

\textbf{Nota}: Esta foi uma funcionalidade assumida, pelo que poderá nem sempre ter sucesso e fazer as corretas associações.

Assim as expressões regulares correspondentes a localizações/países/cidades foram definidas na forma:

\begin{itemize}
\item paises: \\
	\verb!    {enamex}{loc}{subcountry}{pal}{eclose}!
\item ponto final no contexto "country":\\
	\verb!    <country>!
\item cidades no contexto "country":\\
	\verb!    <country>{enamex}{loc}{subcity}{pal}{eclose}!
\end{itemize}

sendo que as abreviaturas mencionadas correspondem a:

\begin{itemize}
\item Elemento "Country"\\
	 \verb!    {subcountry}  (?i:subtype)=\"[ \t]*(?i:country)[ \t]*\"[ \t]*\>!
\item Elemento "City"\\
	 \verb!    {subcity} (?i:subtype)=\"[ \t]*(?i:city)[ \t]*\"[ \t]*\>!
     
\end{itemize}

Estas expressões devem encontrar-se no topo de todas as outras, devido à precedência que possuem sobre elas.

\subsection{Funcionamento}
\label{seq:enamex-func}

No cabeçalho do ficheiro flex são declarados todos os apontadores referentes às estruturas onde irá ser armazenada a informação. Existe um apontador para cada tipo de estrutura, nomeadamente: pessoas, países, cidades, organizações e outras localizações. São também declarados dois apontadores para arrays de caracteres que irão auxiliar o processamento das expressões capturadas. Após início do programa, é efetuada a chamada ao analisador léxico, responsável por capturar os dados referentes às expressões definidas. Cada vez que este efetua uma captura, estes dados são processados, sendo inseridos na estrutura correspondente, sendo que caso necessário lhes são retirados os espaços em branco que possam ter antes e depois da seu conteúdo, por forma a evitar inconsistência de dados. Após o término da leitura do input em questão, todos os dados presentes nas estruturas são escritos nos ficheiros HTML correspondentes, estando estes interligados através de hiperligações (tags <a href=""/>). Cada estrutura (tipo de entidade) é escrita em ficheiro através da função treeToHTML (implementada no ficheiro tree.c), responsável por receber como parâmetros um apontador para estrutura e um identificador de ficheiro (previamente declarado), e transferir a informação para o ficheiro no formato adequado.

\begin{figure}[H]
\centering
\includegraphics[width=7cm]{anexos/2-2/programa.png}
\caption{Componentes do programa}
\end{figure}
\section{Testes realizados}
\label{seq:enamex-test}
Estão documentados neste secção 3 testes realizados ao autómato, utilizando como input os ficheiros que se encontram em anexo: \ref{seq:anex-enamex-test-in01}, \ref{seq:anex-enamex-test-in02} e \ref{seq:anex-enamex-test-in03}.

\subsubsection{Teste nº 1}

Teste focado na funcionalidade de associar diversas cidades a um dado país. Como é possível verificar em anexo, o resultado é uma conexão entre "Portugal" e algumas das suas cidades.

\subsubsection{Teste nº 2}

Teste exemplificado no enunciado. Processado de acordo com o esperado (exemplo da página referente aos locais).

\subsubsection{Teste nº 3}

Teste obtido de um caso real, e focado nas tags de organizações. Resultado correto.


% respectivo enunciado da descricao do problema, das decisoes que lideraram o desenho da solucao e sua implementacao  (incluir a especificacao Flex , deverao conter exemplos de utilizacao (textos fontes diversos e respectivo resultado produzido).

% conceção e arquitetura do sistema

% testes antes da conclusao e intercalar imagens com o texto

% código em apendice em anexo

% dificuldades e decisões antes do código

% imagens bem descritas por texto

% Problema->decisão->solução

% analise e especificação é depois da descrição do problema

% análise: pega nisto, faz isto
% (descascar o problema)=> resultado(especificação) temos isto e dá isto


% respectivo enunciado da descricao do problema, => A introdução depois do titulo 
% as decisoes que lideraram o desenho da solucao e sua implementacao  => estrutura de dados
% (incluir a especificacao Flex ,                       => filtro de texto
% deverao conter exemplos de utilizacao (textos fontes diversos e respectivo resultado produzido) => exemplos
\chapter{Processamento de ficheiros com Canções}
\label{cap:music}
Neste problema era pretendido que fosse criado um filtro de texto que interpretasse ficheiros com letras de musica, e fosse gerado um ficheiro \emph{latex} para cada musica encontrada.
Ainda existe a particularidade de cada ficheiro com musicas poder conter mais do que uma musica, neste caso deve ser criado 2 ficheiros \emph{latex}.

\section{Analise e Especificação}
\label{seq:music-ana}
Existem varias questões que são deixadas em aberto no enunciado que iram ser especificadas nesta secção. 
O programa irá ler do \emph{stardard} input e os nomes dos ficheiros \emph{latex} que irão ser gerados podem ser recebidos por argumento, caso contrario os nomes assumido utilizam numeração, começando em 0 até à n-ésima musica interpretada. 
Uma vez que não se sabe a ordem pela qual os cabeçalhos estão nos ficheiros a ser interpretados, o mais seguro será guardar toda a musica em memoria e só imprimir para o ficheiro \emph{latex} depois do fim da musica.
Após a analise dos \emph{Datasets} verificou-se a existência de campos no cabeçalho que não são utilizados por o programa, ou seja todos os possíveis campos no cabeçalho devem ser ignorados.
Durante a analise também foi verificada a existência de anotações em algumas musicas que serviriam para apresentar as pautas, o nosso programa irá tentar ignorar as marcas e assim tentar apenas imprimir a letra da musica do ficheiro \emph{latex}.
Existe ainda outro cuidado na criação do ficheiro \emph{latex} que é a utilização na musica de caracteres especiais no \emph{latex}.

\section{Implementação}
\label{seq:music-imp}

\subsection{Estrutura de dados}
\label{seq:music-est}
De forma a complementar o enunciado na secção \ref{seq:music-ana}, foi criada uma estrutura principal chamada \verb!Music!, onde se guarda a informação geral da musica temporariamente até esta ser imprimida para um ficheiro.

Nesta estrutura irá-se guardar o titulo, o nome do autor entre outros campos do cabeçalho necessários, e também a letra da musica.

A letra da musica é guardada numa lista ligada onde cada nodo é um linha da letra e é representada pela estrutura \verb!MusicLine!.
A estrutura pode ser encontrada em anexo \ref{seq:anex-music-est}.

\subsection{Filtro de Texto}
\label{seq:music-filtro}
Para a filtragem do texto foram criadas varias expressões regulares, o ficheiro pode ser encontrado em anexo (\ref{seq:anex-music-filtro}).

As primeiras expressões regulares, do tipo \verb!^title:.+! servem para apanhar os cabeçalhos que serão necessários, para alem do \verb!title! existe mais as seguinte: \verb!from!, \verb!author!, \verb!lyrics!, \verb!music! e \verb!singer!, todas com equivalentes. 
De forma a ignorar qualquer outro campo do cabeçalho que não tivesse sido previsto foi ainda criada a seguinte expressão regular: \verb!^[a-zA-Z]+:.+!.

Quanto à detenção da letra da musica existem duas expressões regulares: uma para apanhar uma linha da lírica, outra para apanhar as linhas em branco entre os poemas, que são respetivamente: \verb![ ].*! e \verb!$^\n!.

Tal como dito na analise (\ref{seq:music-ana}), existem algumas anotações no meio da letra da musica que eram necessárias ser retiradas. Para isso foram criadas as seguintes expressões regulares:

\begin{itemize}
\item \verb!{abc}(.|\n)*{abcclose}! para retirar a pauta da musica.
\item \verb![ ].*! que ignora as notas no meio dos poemas (pois estas tem um espaço no inicio).
\end{itemize}

Ainda assim estas duas expressões regulares não eram suficientes e na deteção de uma linha da letra, antes de guardar a linha, passa-se a linha por duas funções: \verb!takeOffAnotations! e \verb!takeOffUnderSccore!. Em que a primeira tira anotações que estao na mesma linha, e a segunda tira os caracteres '\_'  que estão no meio da linha.

\subsection{Funcionamento}
\label{seq:music-func}

De forma a perceber melhor o funcionamento do autómato esta secção irá fazer a ponte entre o filtro de texto (\ref{seq:music-filtro}) e a estrutura de dados (\ref{seq:music-est}).

À medida que o autómato recolhe os campos do cabeçalho da musica, guarda a informação, com as funções de \verb!append!. Como por exemplo \verb!appendAuthor!, \verb!appendLyrics!, entre outras. Estas funções guardam os campos na variável \verb!Music!.

Enquanto que as linhas da letra são guardadas através das funções \verb!appendLine! e \verb!appendWhiteLine!.

Quando é detetado o inicio de uma nova letra, através de expressão regular, é executado \verb!commitCheckNext()! que escreve a letra que esta atualmente na variável \verb!Music! para o ficheiro \emph{latex}, neste ponto caso seja detetado a falta de algum item obrigatório então a escrita para o ficheiro é cancelada.
De seguida a variável \verb!Music! é reiniciada para a musica seguinte com a função \verb!Start()!.

Na escrita do ficheiro \emph{latex} a letra é escrita entre as \emph{tags} da \emph{latex} de \emph{Verbatim} para evitar erros no \emph{latex} por falta de caracteres escape.

\subsection{Testes realizados}
\label{seq:music-test}
Estão documentados neste secção 3 testes realizados ao autómato, utilizando com input os ficheiro que estão em anexo: \ref{seq:anex-music-test-in01}, \ref{seq:anex-music-test-in02} e \ref{seq:anex-music-test-in03}.

\subsubsection{Teste nº 1}

Após a utilização do autómato no ficheiro \ref{seq:anex-music-test-in01}, este gerou o output (\ref{seq:anex-music-test-out01}).
Este ficheiro não tem nenhuma situação excecional, é um caso normal.

\subsubsection{Teste nº 2}

Após a utilização do autómato no ficheiro \ref{seq:anex-music-test-in02}, este gerou o output (\ref{seq:anex-music-test-out02}).
Este ficheiro tem duas situações excecionais, o carácter '\_' no meio de palavras e notas musicais no fim das frases.
Podemos verificar no output que apenas tem a letra da musica.

\subsubsection{Teste nº 3}

Após a utilização do autómato no ficheiro \ref{seq:anex-music-test-in03}, este gerou o output (\ref{seq:anex-music-test-out03}).
Este ficheiro tem uma situação excecional, antes da letra da musica tem as tags \verb!<abc>...</abc>! com anotações de notas musicas.
Podemos verificar que no output já não está presente.


Um exemplo de uma possível utilização do autómato é:\\
\verb!cat <inputFile> | ./play <output1.tex> <output2.tex> ...!


\chapter{Anexos}
\label{cap:anex}


\section{Museu da Pessoa — tratamento de fotografias}
\label{seq:anex-museu}


\subsection{Filtro de Texto}
\label{seq:anex-museu-filtro}


\subsection{Estrutura de dados}
\label{seq:anex-museu-est}


\subsection{Cabeçalho ficheiro C}
\label{seq:anex-museu-header}


\section{Processamento de Entidades Nomeadas (Enamex)}
\label{seq:anex-enamex}


\subsection{Filtro de Texto}
\label{seq:anex-enamex-filtro}


\subsection{Estrutura de dados}
\label{seq:anex-enamex-est}


\subsection{Cabeçalho ficheiro C}
\label{seq:anex-enamex-header}


\section{Processamento de ficheiros com Canções}
\label{seq:anex-music}


\subsection{Filtro de Texto}
\label{seq:anex-music-filtro}
\verbatiminput{2-5.flex}


\subsection{Estrutura de dados}
\label{seq:anex-music-est}
\begin{verbatim}
typedef struct sMusicLine {
    char* line;
    struct sMusicLine* next;
} MusicLine;

typedef struct sMusic {
    char* _Title;
    char* _From;
    char* _Author;
    char* _Lyrics;
    char* _Music;
    char* _Singer;

    MusicLine* poem;
    MusicLine* poemEnd;
    int error;
} Music;
\end{verbatim}


\subsection{Cabeçalho ficheiro C}
\label{seq:anex-music-header}
\verbatiminput{2-5.h}

\subsection{Testes}
\label{seq:anex-music-tests}
\subsubsection{Input teste 1}
\label{seq:anex-music-test-in01}
\verbatiminput{anexos/2-5-a-in}

\subsubsection{Output teste 1}
\label{seq:anex-music-test-out01}
\verbatiminput{anexos/2-5-a-out}

\subsubsection{Input teste 2}
\label{seq:anex-music-test-in02}
\verbatiminput{anexos/2-5-a-in}

\subsubsection{Output teste 2}
\label{seq:anex-music-test-out02}
\verbatiminput{anexos/2-5-b-out}

\subsubsection{Input teste 3}
\label{seq:anex-music-test-in03}
\verbatiminput{anexos/2-5-c-in}

\subsubsection{Output teste 3}
\label{seq:anex-music-test-out03}
\verbatiminput{anexos/2-5-c-out}

\begin{figure}
\centering
\includegraphics[width=15cm]{anexos/2-5-a-img1.png}
\caption{PDF gerado por o ficheiro latex (teste 1). Pagina 1 de 2}
\end{figure}

\begin{figure}
\includegraphics[width=15cm]{anexos/2-5-a-img2.png}
\caption{PDF gerado por o ficheiro latex (teste 1). Pagina 2 de 2}
\label{fig::anex-music-test-img}
\end{figure}

\begin{figure}
\centering
\includegraphics[width=15cm]{anexos/2-5-b-img.png}
\caption{PDF gerado por o ficheiro latex (teste 2). Pagina 1 de 2}
\label{fig::anex-music-test-img02}
\end{figure}

\begin{figure}
\includegraphics[width=15cm]{anexos/2-5-b-img2.png}
\caption{PDF gerado por o ficheiro latex (teste 2). Pagina 2 de 2}
\label{fig::anex-music-test-img}
\end{figure}


\begin{figure}
\centering
\includegraphics[width=15cm]{anexos/2-5-c-img.png}
\caption{PDF gerado por o ficheiro latex (teste 3)}
\label{fig::anex-music-test-img03}
\end{figure}
\end{document}


% respectivo enunciado, da descricao do problema, das decisoes que lideraram o desenho da solucao e sua implementacao (incluir a especificacao Flex , deverao conter exemplos de utilizacao (textos fontes diversos e respectivo resultado produzido)